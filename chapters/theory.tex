\chapter{Theory}

In the following sections, examples of a figure, an equation, a table, a chemical structure, a list, a listing and a to-do note are shown.
These examples are not particularly well-made\dots

\section{Figure}
\begin{figure}[H]
    \centering
    \includegraphics[width=0.45\linewidth, trim=3cm 11cm 3cm 11cm]{chapters/theory/X.pdf}
    \includegraphics[width=0.45\linewidth, trim=3cm 11cm 3cm 11cm]{chapters/theory/Y.pdf}
    \caption{Surface and contour plots showing the two dimensional function $z(x,y)=\sin(x+y)\cos(2x)$.}
\end{figure}

\section{Equation}
\begin{equation}
    f(t)=\left\{%
    \begin{array}{ll}
        1,\qquad & t< 1 \\
        t^2 & t\geq 1
    \end{array}\right.
\end{equation}

\section{Table}
\begin{table}[H]
    \centering
    \caption[This (instead of the potentially long caption) appears in the list of tables.]{Values of $f(t)$ for $t=0,1,\dots 5$.}
    \begin{tabular}{lllllll}
        \toprule
        $t$ & 0 & 1 & 2 & 3 & 4 & 5 \\ \midrule
        $f(t)$ & 1 & 1 & 4 & 9 & 16 & 25 \\ \bottomrule
    \end{tabular}
\end{table}

\section{Chemical structure}
\begin{center}
    \chemfig{X*5(-E-T-A-L-)}
\end{center}

\section{List}
\begin{enumerate}
    \item The first item
    \begin{enumerate}
        \item Nested item 1
        \item Nested item 2
    \end{enumerate}
    \item The second item
    \item The third item
    \item \dots
\end{enumerate}

\section{Source code listing}
An example of a source code listing.
The output can be drastically improved with the optional settings.
\lstset{language=Matlab}
\begin{lstlisting}[frame=single]
% Generate x- and y-nodes
x = linspace(0, 1);
y = linspace(0, 1);

% Calculate z = f(x, y)
for i = 1:length(x)
    for j = 1:length(y)
        z(i, j) = x(i) + 2*y(j);
    end
end
\end{lstlisting}

\section{To-do note}
The \texttt{todo} package enables to-do notes to be added in the page margin. This can be a very convenient way of making notes in the document during the process of writing. All notes can be hidden by using the option \emph{disable} when loading the package in the settings. \todo{Example of a to-do note.}
